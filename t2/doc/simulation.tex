\section{Simulation Analysis}
\label{sec:simulation}
The Ngspice Software was used to simulate the circuit. Take note that $gb$
represents $I_b$ and $id = I_d$, $v(i)$ represents the voltage in node i.
It was used the same Node nomenclature as in sections before with two excpetion: 
Node 9 was added to the circuit between Node 7 and Node 8 in order to sence the current
in that point of the circuit as 0V voltage source and node 4 was named node 0. (Resistor 7 from now on is placed
between Node 8 and Node 9, Virtual 0V voltage source is placed between Node 7 and Node 8).


\subsection{Exercise 1}
In this section we simulate the operating point for $t<0$ in order to obtain the
voltages in all nodes and the currents in all branches.
The table \ref{tab:op1} shows the results.

\begin{table}[ht]
  \centering
  \begin{tabular}{|c|c|}
    \hline    
    {\bf Name} & {\bf Value [A or V]} \\ \hline
    \input{../sim/op_tab1}
  \end{tabular}
  \vspace{20mm}
  \caption{Operating point. A variable preceded by @ is of type {\em current}
    and expressed in Ampere; other variables are of type {\it voltage} and expressed in
    Volt.}
  \label{tab:op1}
\end{table}

\newpage

\subsection{Exercise 2}
Next we simulate the operating point for vs(0)=0, replacing the capacitor with  with a 
voltage source Vx = V(6)-V(8). Take note that V(6) and V(8) are the voltages in nodes 6 and 8 as obtained in Exercise 1.
The reason for making this step is to find the initial conditions for the time-dependent circuit
that will be analysed in the following exercises.
Please note that voltage values with e-14 and e-15 exponent should represent zero (as expected).
Comparing the order of magnitude of this values with the other voltages playing in this circuit it totally neglegible.
The table \ref{tab:op2} shows the results.

\begin{table}[ht]
  \centering
  \begin{tabular}{|c|c|}
    \hline    
    {\bf Name} & {\bf Value [A or V]} \\ \hline
    \input{../sim/op_tab2}
  \end{tabular}
  \vspace{20mm}
  \caption{Operating point. A variable preceded by @ is of type {\em current}
    and expressed in Ampere; other variables are of type {\it voltage} and expressed in
    Volt.}
  \label{tab:op2}
\end{table}

\newpage

\subsection{Exercise 3}
For this exercise we Simulate the natural response of the circuit 
using the boundary conditions V(6) and V(8) as obtained in Exercise 2.
We used Ngpice’s transient analysis mode to get v6(t) in the interval [0, 20] ms.
The figure \ref{fig:trans3} plots the result.

  \begin{figure}[ht] \centering
  \caption{Natural solution, $v_{6n}(t)$ ($V$) in the interval $[0,20]ms$}
  \includegraphics[width=0.6\linewidth]{trans3.pdf}
  \label{fig:trans3}
  \end{figure}
The voltage decays exponentially as expected in theory.
\newpage

\subsection{Exercise 4}
In Exercise 4 we Simulate the natural and forced response on node 6 by repeating Exercise  3 
with vs(t) as given in the introduction of this report and f=1kHz. 
The figure \ref{fig:trans4} shows both the stimulus and the response plots.

  \begin{figure}[ht] \centering
  \caption{Natural solution, $v_{6n}(t)$ and $V$ in the interval $[0,20]ms$}
  \includegraphics[width=0.6\linewidth]{trans4.pdf}
  \label{fig:trans4}
  \end{figure}

\newpage

\subsection{Exercise 5}
Lastly we simulate the frequency response in node 6.
Take note that the frequency is in logscale, magnitude in dB and phase in degrees.
The simulation is performed for the frequency range 0.1 Hz to 1 MHz.
In figure \ref{fig:trans5db} we plot the amplitudes of $V_{6}$,$V_{s}$ and $V_{c}$ in dB
as function of the frequency.
These values are represented respectively as db(v(6)), db(v(1)) and db(v(6)-v(8)).

On the other side, in figure \ref{fig:trans5ph} we plot the phase of $V_{6}$,$V_{s}$ and $V_{c}$ in degrees
as function of the frequency.
These values are represented respectively as ph(v(6)), ph(v(1)) and ph(v(6)-v(8)).


  \begin{figure}[ht] \centering
  \caption{Amplitudes in dB of $V_{6}$,$V{s}$ and $V_{c}$ along the frequency range: 0.1Hz to 1 MHz }
  \includegraphics[width=0.6\linewidth]{trans5db.pdf}
  \label{fig:trans5db}
  \end{figure}

  $V_{s}$ is always zero because it's the voltage source and it keeps its amplitude constant along
  all the frequencies. The value is zero because the units are presented in dB, the base 10 
  logarithm of the amplitude in Volts, i.e. it corresponds to 1V. The reason for the evolution of the other curves is already
  explained in section \ref{sec:6} of the theoretical analysis.

  \newpage

  \begin{figure}[ht] \centering
  \caption{Phase in degrees of $V_{6}$,$V{s}$ and $V_{c}$ along the frequency range: 0.1Hz to 1 MHz }
  \includegraphics[width=0.8\linewidth]{trans5ph.pdf}
  \label{fig:trans5ph}
  \end{figure}

  As ploted, all of the phases start at 0 degrees with the frequency value being 0.1Hz.
  The reason for the evolution of the other curves is already explained in section \ref{sec:6} of the theoretical analysis.

\newpage
%\begin{table}[h]
 % \centering
  %\begin{tabular}{|c|c|}
   % \hline    
    %{\bf Name} & {\bf Value [A or V]} \\ \hline
    %\input{../sim/op_tab}
%  \end{tabular}
%  \vspace{20mm}
%  \caption{Operating point. A variable preceded by @ is of type {\em current}
%    and expressed in Ampere; other variables are of type {\it voltage} and expressed in
%    Volt.}
%  \label{tab:op}
%\end{table}







