\section{Introduction}
\label{sec:introduction}

% state the learning objective 
The objective of this laboratory assignment is to study a circuit with various resistors and a capacitor. Besides the RC part of the circuit, we were also 
capable of studying the circuit evolution due to a sinosoidal voltage source, adding to a current controlled voltage source and a voltage controlled current source, 
all represented in fig. \ref{fig:rc}. \\
Using the tool Octave for a theoretical analysis of the circuit and creating a computed simulation with the tool Ngpice we were able to compare the data obtained by 
studying this complex circuit.

Because of the sinosoidal voltage source,

\vspace{-13mm}
\begin{multicols}{2}
    \begin{equation}
        v_s(t) = V_s \times u(-t) + sin(2\pi ft) \times u(t)
    \end{equation}\break
    \begin{equation}
        u(t) = \begin{cases} 0 & t < 0 \\ 1 & t \ge 0 \end{cases}
    \end{equation}
\end{multicols}

various time stamps were evaluated.
Resorting to Ohm's Law and the Kirchhoff Current Law (KCL) for the various solutions in time it was possible to get all the information contained in the circuit. 

In Section~\ref{sec:analysis}, a theoretical analysis of the circuit is presented, explaining every point. 
In Section~\ref{sec:simulation}, the circuit is analysed by simulation, and the results are compared to the theoretical results obtained in
Section~\ref{sec:analysis}. The conclusions of this study are outlined in Section~\ref{sec:conclusion}.


\begin{figure}[h] 
    \centering
    \includegraphics[width=0.85\linewidth]{../doc/rc.pdf}
    \caption{Representation of the circuit analysed}
    \label{fig:rc}
\end{figure}
    

\begin{table}[h]
    \centering
    \begin{tabular}{c|c|c|c|c}  
        \hline
        \input{../mat/datatab1.tex}
    \end{tabular}
    \begin{tabular}{c|c|c|c|c|c}
        \hline
        \input{../mat/datatab2.tex} 
    \end{tabular}
    \caption{Given Data Rounded Values}
    \label{tab:data}
\end{table}

