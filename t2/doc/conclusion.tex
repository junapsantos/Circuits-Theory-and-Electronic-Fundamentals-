\vspace{-5mm}
\section{Values and comparison}


\begin{table}[h]
    \vspace{-5mm}
    \parbox{.45\linewidth}{
    \centering
    \begin{tabular}{|l|r|}
    \hline
    {\bf Name} & {\bf Value } \\ \hline
    \input{../sim/op_tab1.tex}
    \end{tabular}
    \vspace{-2mm}
    \caption{Simulated Operating Point for $t < 0$}
    \label{ngspice}
    }
    \hfill
    \parbox{.45\linewidth}{
    \centering
    \begin{tabular}{|l|r|}
        \hline
        {\bf Name} & {\bf Value } \\ \hline
        \input{../mat/node1.tex}
    \end{tabular}
    \vspace{-2mm}
    \caption{Theoretical Operating Point for $t < 0$}
    \label{analise teorica - ponto 1}
    }
\end{table}


\begin{table}[h]
    \vspace{-5mm}
    \parbox{.45\linewidth}{
    \centering
    \begin{tabular}{|l|r|}
    \hline
    {\bf Name} & {\bf Value } \\ \hline
    \input{../sim/op_tab2.tex}
    \end{tabular}
    \vspace{-2mm}
    \caption{Simulated Operating Point for $t = 0$}
    \label{ngspice}
    }
    \hfill
    \parbox{.45\linewidth}{
    \centering
    \begin{tabular}{|l|r|}
        \hline
        {\bf Name} & {\bf Value } \\ \hline
        \input{../mat/node2.tex}
    \end{tabular}
    \vspace{-2mm}
    \caption{Theoretical Operating Point for $t = 0$}
    \label{analise teorica - ponto 2}
    }
\end{table}

\vspace{-2mm}
Considering the exponents -18 and -15 as zero, the values from both tables, obtained with ngspice and octave, are equal, considering only some difference in the signals
of some voltages and currents explained by the nodal direction chosen in the declaration of the components in ngspice.
Every significant figure of ngspice presented in the tables is in order with the numbers obtained from octave, except as mentioned before, the cases where they equal zero.


\newpage

\section{Conclusion}
\label{sec:conclusion}
\vspace{10mm}
In this laboratory assignment the objective of analysing a RC circuit has been achieved. 
Static, time and frequency analyses have been performed both
theoretically using the Octave maths tool and by circuit simulation using the
Ngspice tool.
The results obtained in the Theoretical section by octave in both the nodal and mesh theoretical analysis matched each other. 
The simulation results matched the theoretical results very closely too.
Graphically there's no distinction between the plots made in ngspice and octave except GUI properties.
There are some small numerical differences in table results comparing again octave with ngspice, but thats due to the fact that
the number of decimal places considered by this two programs is different.
We should emphasize that it was crucial to use the theoretical tool of Equivalent Resistor from the Thevenin's Theorem
to drastically simplify our study of the circuit behaviour.

