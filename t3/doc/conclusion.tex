\vspace{-5mm}
\section{Values and comparison}


The value obtained for the exit DC Voltage (tab. \ref{tab:teovalues}) differs from the one obtained in the simulation 
(tab. \ref{tab:simvalues}) by only $9.71 mV$. Although we used the same number of diodes and the same resistance value for
 the resistor $R$, we still weren't able to perfectly synchronize the two. Since the model of the diodes is different in 
 the two programs, that must be the reason for the difference between the voltage regulator effect. The major difference
 is between the ripple in both theoretical and simulation analysis and this is due to the approximation explained in section
  \ref{subsec:2} when calculating the value for the ripple.

 \vspace{5mm}
 \begin{table}[h]
    %\vspace{-5mm}
    \parbox{.45\linewidth}{
    \centering
    \begin{tabular}{|c|c|}
        \hline    
        {\bf Quantities} & {\bf Volts / MU} \\ \hline
        \input{../sim/values_tab.tex}
    \end{tabular}
    \vspace{-2mm}
    \caption{{\bf{Simulation}} Average Voltage Value of DC Output (V). Deviation value (V). Ripple Value (Max-Min) (V). Cost of Circuit (MU). Merit of Circuit.}
    \label{ngspice}
    }
    \hfill
    \parbox{.45\linewidth}{
    \centering
    \begin{tabular}{|c|c|}
        \hline
      {\bf Quantaties} & {\bf Volts} \\ \hline 
average & 1.200972e+01 \\ \hline 
ripple & 8.990121e-02 \\ \hline 
 deviation & 9.718317e-03 \\ \hline 

    \end{tabular}
    \vspace{-2mm}
    \caption{{\bf{Theoretical}} Value of DC Output (V). Ripple Value (V). Deviation value (V). }
    \label{analise teorica - ponto 2}
    }
\end{table}

\section{Conclusion}
\label{sec:conclusion}
\vspace{10mm}
In this laboratory assignment the objective of simulating an ACDC converter has been achieved. 
Time analysis has been performed both
theoretically using the Octave maths tool and by circuit simulation using the
Ngspice tool.
The results obtained in Octave and Ngspice matched each other with a considerable error. 
Graphically there are little distinctions between the plots made in ngspice and octave.
The differences are due to the fact that
the diode models weren't compatible in the two programs.
In order to solve the complex non linear equations of the theoretical analysis some aproximations were made,
that were explained in the Analysis Section.
We also should note that in ngspice simulation we had to made the transient analysis starting at 10 seconds
so we could observe only the stationary part.
Right now we are able to build our own AC adapter to charge our devices.
