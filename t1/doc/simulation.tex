\section{Simulation Analysis}
\label{sec:simulation}
\vspace{10mm}
\subsection{Operating Point Analysis}
The Ngspice Software was used to simulate the circuit. Considering the simplicity of the circuit only
Operating Point Analysis was performed.
There was the need to add a virtual voltage source which
would sense the controlling current $I_c$. For that reason, node number 7 was virtually created between node 6 and node 8.
Table~\ref{tab:op} shows the simulated operating point results for the circuit
under analysis. As it can be seen, there are no differences compared to the theoretical analysis results. Take note that $gb$
represents $I_b$ and $id = I_d$, $v(i)$ represents the voltage in node i. Once there's an extra node, node labelling has been changed.

The nodes are numbered according to figure~\ref{fig:rcsim}. The plus and minus signs in the resistors terminals
are used to evidence the current direction choosen for each branch, i.e., it's assumed a priori 
the direction of the current pointing from plus to minus sign. For the other components it's used the current direction
according to the arrow presented or the plus/minus signs already drawn in the figure~\ref{fig:rc}. 


\vspace{10mm}

\begin{table}[ht]
  \centering
  \begin{tabular}{|c|c|}
    \hline    
    {\bf Name} & {\bf Value [A or V]} \\ \hline
    \input{../sim/op_tab}
  \end{tabular}
  \vspace{20mm}
  \caption{Operating point. A variable preceded by @ is of type {\em current}
    and expressed in Ampere; other variables are of type {\it voltage} and expressed in
    Volt.}
  \label{tab:op}
\end{table}



\begin{figure}[ht] \centering
  \includegraphics[width=1.\linewidth]{rcsim.pdf}
  \caption{Ngspice circuit scheme representation}
  \label{fig:rcsim}
\end{figure}





