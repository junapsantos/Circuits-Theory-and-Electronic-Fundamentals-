\section{Introduction}
\label{sec:introduction}

% state the learning objective 
The objective of this laboratory assignment is to study a circuit from three different methods and compare the outcomes. Those are: mesh analysis, node analysis and ngspice simulation.
The circuit can be seen if Figure 1.
Resorting to Ohm's Law, Kirchhoff Current Law (KCL) and Kirchhoff Voltage Law (KVL) it was possible to get all the information contained in the circuit. 

From KCL we know that the sum of currents converging (diverging) in
a node is null

\[ \sum_{i=1}^{n} I\textsubscript{i} = 0 \]

From KVL we deduce that the sum of Voltages
in a circuit loop is null
\[ \sum_{i=1}^{n} V\textsubscript{i} = 0 \]

In Section~\ref{sec:analysis}, a theoretical analysis of the circuit is
presented. In Section~\ref{sec:simulation}, the circuit is analysed by
simulation, and the results are compared to the theoretical results obtained in
Section~\ref{sec:analysis}. The conclusions of this study are outlined in
Section~\ref{sec:conclusion}.

\begin{figure}[h] \centering
\includegraphics[width=0.85\linewidth]{rc.pdf}
\caption{Lab01 Circuit}
\label{fig:rc}
\end{figure}


\begin{table}[h]
    \centering
    \begin{tabular}{c|c|c|c|c}
        \hline
        R1 (kOhm)& R2 (kOhm)& R3 (kOhm)& R4 (kOhm)& R5 (kOhm) \\
        1.01072 & 2.0468 & 3.05917 & 4.1965 & 3.07880     
    \end{tabular}
    \begin{tabular}{c|c|c|c|c|c}
        \hline
        R6 (kOhm)& R7 (kOhm)& Va (V)& Id (mA)& Kb (mS)& Kc (kOhm)\\
        2.08426 & 1.01420 & 5.03865 & 1.01597 & 7.14517 & 8.13498 \\
        \hline  
    \end{tabular}
    \caption{Given Data Rounded Values}
    \label{tab:data}
\end{table}

