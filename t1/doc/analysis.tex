\section{Theoretical Analysis}
\label{sec:analysis}

In this section, the circuit shown in Figure~\ref{fig:rc} is analysed
theoretically in two different ways: Mesh Analysis and Node Analysis. After obtaining circuit values, to confirm the coherence between the results from these two analysis,
the voltage and current values were also calculated using Ohm's Law $R = \frac{V}{I}$ and the definition of current/voltage controled voltage/current sources: 
$V_c = K_c \times I_c$ and $I_b = K_b \times V_b$, repectively.


\subsection{Mesh Analysis}
This type of circuit analysis implies that we atribuite a current (with arbitrary direction) to each elementar mesh. 
Since our circuit has four elementar meshes, we assingned $I_a$, $I_b$, $I_c$ and $I_d$ as depicted in Figure \ref{fig:rc}. 
For the last three currents, the directions were specially choosen so that they had the same signal as the currents from the 
current generators (b and d) and the scheme given. \\
Since the value of $I_d$ was known, we found 3 linaerly independent equations to find the values of the missing currents 
and derived the following matrix:

\begin{gather}
    \begin{bmatrix}
       R_1 + R_4 + R_3 & R_3 & R_4 \\
       R_4 & 0 & R_6 + R_7 + R_4 - K_c \\
       K_b \times R_3 & K_b \times R_3 - 1 & 0 \\
    \end{bmatrix}
    \begin{bmatrix}
        I_a \\
        I_b \\
        I_c \\
    \end{bmatrix}
    =
    \begin{bmatrix}
        V_a \\
        0 \\
        0 \\
    \end{bmatrix}
\end{gather}

The first two lines of the matrix are directly derived from analysing the path of the current in the respective elementar mesh. 
The third line comes from finding a relation between the currents $I_b$ and $I_a$,  substituting voltage $V_b$ that goes through the resistor 3:
\begin{equation}
    I_b = K_b \times V_b = \Leftrightarrow I_b = K_b \times R_3 \times (I_a + I_b) \Leftrightarrow I_a(K_b R_3) + I_b(K_b R_3 -1) = 0
\end{equation}

This system of linear equations was solved using the tool Octave and the values obtained can be consulted in Table \ref{tab:oct}.


\subsection{Node Analysis}
The Node Analysis was based on Kirchhoff Current Law (KCL) and Ohm Law.
From node 1 to  4 and 6 to 7 KCL was applied and resulted in an equation for each node.

In order to obtain the equations of nodes 5 and 8, since between them there is a current-controlled voltage source, it isn't possible to write the node equation from KCL and we are facing a special case in this node analysis. 

This case was solved with the help of Supernode analysis. This theoretical concept considers two nodes as just one, in between those nodes it must be placed a voltage source. In this particular case, we first conclude that the voltage difference between node 5 and node 8 is V\textsubscript{c}. Then, we consider nodes 5 and 8 as just one node and apply Kirchhoff Current Law since the current that leaves node 5 is the current that arrives at 8 hence the total current in the new node is null. Now, we are able to apply the KCL to the supernode. 

From the supernode analysis, we were able to find two equations

\begin{equation}
V5 - V8 = V\textsubscript{c} 
\end{equation}

in which Vc can be rewritten as 

\begin{equation}
V\textsubscript{c} = Kc\times Ic = Kc\times (V7-V6)/R6
\end{equation}

In addition, by applying KCL to the supernode we get

\begin{equation}
\frac{V5 - V6}{R4} + \frac{V5 - V2}{R3} + \frac{V5 - V4}{R5} + \frac{V8 - V7}{R7} - Id = 0  
\end{equation}

Given that the ground was placed at node 6 the voltage in that node is null (V6 = 0 V).

The following matrix system has the information of the equations in all the nodes and its resolution leads us to the voltage value in each node (Table \ref{tab:oct}). 

Gi represents 1/Ri and Vi is the voltage in node i.

\begin{gather}
    \begin{bmatrix}
        1       &   0       & 0     &   0 & 0 &0 & 0. \\
        -G1      &   G1+G2+G3       & -G2     &   0 & -G3 &0 & 0  \\
        0       &   -G2-Kb       & G2     &   0 & Kb &0 & 0  \\   
        0 & Kb & 0 & G5 & -Kb-G5 & 0 & 0  \\
        0       &   0       & 0     &   0 & 1 & Kc \times G6 & -1  \\
        0       &   0       & 0     &   0 & 0 & G6+G7 & -G7  \\
        0 & -G3 & 0 & -G5 & G3+G4+G5 & -G7&G7 \\
    \end{bmatrix}
    \begin{bmatrix}
        V1     \\
        V2     \\
        V3     \\
        V4     \\
        V5     \\
        V7     \\
        V8     \\
    \end{bmatrix}
    =
    \begin{bmatrix}
        Va     \\
        0      \\
        0      \\
        Id     \\
        0      \\
        0      \\
        -Id    \\
    \end{bmatrix}
\end{gather}

\subsection{Values and Comparison}

\begin{table}[h]
    \centering
    \begin{tabular}{|l|r|}
      \hline    
      {\bf Name} & {\bf Value [A or V]} \\ \hline
      Ia & 1.936080e-04 \\ \hline 
Ib & -2.028901e-04 \\ \hline 
Ic & 9.672069e-04 \\ \hline 
Id & 1.015970e-03 \\ \hline 
V1 & 5.038651e+00 \\ \hline 
V2 & 4.842967e+00 \\ \hline 
V3 & 4.427687e+00 \\ \hline 
V4 & 8.623984e+00 \\ \hline 
V5 & 4.871363e+00 \\ \hline 
V6 & 0.000000e+00 \\ \hline 
V7 & -2.015911e+00 \\ \hline 
V8 & -2.996850e+00 \\ \hline 

    \end{tabular}
    \caption{Table with values of current (A) and voltage (V) obtained from mesh and node analysis of the circuit using the tool Octave.}
    \label{tab:oct}
\end{table}


Calculating voltages using the currents from mesh analysis to compare with node analysis, we get:

\begin{equation}
    V_b = R_3 \times (I_a+I_b) = \frac{I_b}{K_b} = -0.0283956 V = V_2 - V_5 (Table \ref{tab:oct})
\end{equation}

\begin{equation}
    V_c = K_c \times I_c = 7.868213 V = V_5 - V_7 (Table \ref{tab:oct})
\end{equation}


In order to compare the voltage values to the mesh analysis, the currents were determined by the following equations

\begin{equation}
I\textsubscript{a} = \frac{V1 - V2}{R1} = 1.936\times 10\textsuperscript{-4} A
\end{equation}

\begin{equation}
I\textsubscript{b} = Kb \times (V2-V5) = -2.0289\times 10\textsuperscript{-4} A
\end{equation}

\begin{equation}
I\textsubscript{c} = \frac{V7}{R6} = 9.6721 \times 10\textsuperscript{-4} A
\end{equation}

These calculations make it possible to conclude that the two analysis are equivelent and lead to the same results.